

% RECOMMENDED %%%%%%%%%%%%%%%%%%%%%%%%%%%%%%%%%%%%%%%%%%%%%%%%%%%
\documentclass[graybox]{svmult}

% choose options for [] as required from the list
% in the Reference Guide

%\usepackage{type1cm}        % activate if the above 3 fonts are
%                            % not available on your system
%
%\usepackage{makeidx}         % allows index generation
\usepackage{graphicx}        % standard LaTeX graphics tool
                             % when including figure files
\usepackage{multicol}        % used for the two-column index
\usepackage[bottom]{footmisc}% places footnotes at page bottom
\usepackage{framed} % vince may 17


\usepackage{newtxtext}       % 
\usepackage[varvw]{newtxmath}       % selects Times Roman as basic font

% see the list of further useful packages
% in the Reference Guide

%\makeindex             % used for the subject index
                       % please use the style svind.ist with
                       % your makeindex program

%%%%%%%%%%%%%%%%%%%%%%%%%%%%%%%%%%%%%%%%%%%%%%%%%%%%%%%%%%%%%%%%%%%%%%%%%%%%%%%%%%%%%%%%%

\begin{document}

\title*{Bioconductor's Computational Ecosystem for Genomic Data Science in Cancer}
\titlerunning{Bioconductor for Cancer Data Science}
% Use \titlerunning{Short Title} for an abbreviated version of
% your contribution title if the original one is too long
\author{Name of First Author\orcidID{0000-1111-2222-3333} and\\ Name of Second Author\orcidID{1111-2222-3333-4444}}
% Use \authorrunning{Short Title} for an abbreviated version of
% your contribution title if the original one is too long
\institute{Name of First Author \at Name, Address of Institute, \email{name@email.address}
\and Name of Second Author \at Name, Address of Institute \email{name@email.address}}
%
% Use the package "url.sty" to avoid
% problems with special characters
% used in your e-mail or web address
%
\maketitle

\abstract{
The Bioconductor project enters its third decade with over two 
thousand packages for genomic data science, over 100,000 annotation and 
experiment resources, and a global system for convenient distribution to 
researchers. Over 60,000 PubMed Central citations and terabytes of content 
shipped per month attest to the impact of the project 
on cancer genomic data science. This report provides an overview 
of cancer genomics resources in Bioconductor. After an overview 
of Bioconductor project principles, we address exploration 
of institutionally curated cancer genomics data such as TCGA. 
We then review genomic annotation and ontology resources 
relevant to cancer and then briefly survey analytical 
workflows addressing specific topics in cancer genomics. 
Concluding sections cover how new software and data 
resources are brought into the ecosystem and how the 
project is tackling needs for training of the research 
workforce. Bioconductor's strategies for supporting 
methods developers and researchers in cancer genomics 
are evolving along with experimental and computational 
technologies. All the tools described in this report 
are backed by regularly maintained learning resources 
that can be used locally or in cloud computing environments.
}



\section{Introduction}
\label{sec:1}

Computation is a central component of cancer genomics
research. Tumor sequencing is the basis of computational
investigation of mutational, epigenetic and immunologic
processes associated with cancer initiation and progression.
Numerous computational workflows have been produced to
profile tumor cell transcriptomes and proteomes.
New technologies promise to unite sequence-based
characterizations with digital histopathology,
ultimately driving efforts in molecule design
and evaluation to produce patient-centered treatments.

Bioconductor is an open source software project with
a 20 year history of uniting biostatisticians, bioinformaticians,
and genome researchers in the creation of an ecosystem
of data, annotation, and analysis resources for research
in genome-scale biology. This paper will review current
approaches of the project to advancing cancer genomics.
After a brief discussion of basic principles of the Bioconductor
project, we will present a ``top down'' survey of resources
useful for cancer bioinformatics. Primary sections address

\begin{itemize}
\tightlist
\item
  how to explore institutionally curated cancer genomics data
\item
  genomic annotation resources relevant to cancer genomics
\item
  analytical workflows
\item
  components for introducing new data or analyses
\item
  pedagogics and workforce development.
\end{itemize}

Final sections provide enumerations of software and data packages
tagged by their contributors as specifically relevant to cancer.

\section{Bioconductor principles}


\subsection{R packages and vignettes}\label{r-packages-and-vignettes}}

Software tools and data resources in Bioconductor are organized
into ``R packages''. These are collections of folders with data,
code (principally R functions), and documentation
following a protocol specified in
the
Writing R Extensions manual \cite{WRE}.  R packages have a DESCRIPTION file with metadata about
package contents and provenance. Package structure can be
checked for validity using the \texttt{R CMD check} facility.
Documentation of code and data can be programmatically
checked for existence and validity. The DESCRIPTION file
for a package specifies its version and
also gives precise definition of how an R package may
depend upon versions of other packages.

At its inception,
Bioconductor introduced a new approach to holistic package
documentation called ``vignette''.
Vignettes provide narrative and explanation interleaved with
executable code describing package operations.
While R function manual pages describe
the operation of individual functions,
vignettes illustrate the interoperation
of package components and provide motivation
for both package design but also context
for its use.

\subsection{R package repositories; repository evolution}\label{r-package-repositories-repository-evolution}}

Bioconductor software forms a coherent ecosystem that
can be checked for consistency of versions of all
packages available in a given installation of R.
Bioconductor packages may specify dependency on
other Bioconductor packages, or packages that are
available in the CRAN repository. Bioconductor does
not include packages with dependencies on ``github-only''
packages. Later in this paper we will provide details
on package quality assurance that provide a rationale
for this restriction.

Major updates to the R language occur annually, and
updates are preceded by careful assessment of effects of
language change on Bioconductor package operations. These effects
can be identified through changes in the output of R CMD check.
The Bioconductor ecosystem is updated twice a year, once
to coincide with update to R, and once about six months
later. The semianual updates reflect the need to track
developments in the fast-moving field of genomic data science.

\subsection{Package quality assessment; installation consistency}\label{package-quality-assessment-installation-consistency}}

The BiocCheck function is used to provide more
stringent assessment of package compliance with basic
principles of the Bioconductor ecosystem.

The BiocManager package provides for installing and updating package
and has functionality for verifying the coherence and version status
of the currently installed package collection.
This is important
in the context of a language and package ecosystem
that changes every six months, while analyses may
take years to complete. Tools for recreating past
package collections are available to assist in
reproducing outputs of prior analyses.

\subsection{Unifying assay and sample data: SummarizedExperiment and MultiAssayExperiment}\label{unifying-assay-and-sample-data-summarizedexperiment-and-multiassayexperiment}}

Most of the data from genome-scale experiments to be discussed
in this chapter are organized in special data containers
rooted in the concepts of the SummarizedExperiment class.
Briefly, assay data are thought of as occupying a \(G \times N\)
array, and sample level data occupy an \(N \times K\) table. The array
and the table are linked together in the SummarizedExperiment; see Figure \ref{fig:sesc}.

\begin{figure}
\includegraphics[width=0.8\linewidth,]{SEschema} \caption{SummarizedExperiment schematic.}\label{fig:sesc}
\end{figure}

Multiple representations of assay results may be managed in this
structure, but all assay arrays must have dimensions \(G \times N\).

For experiment collections in which the same samples are subjected
to multiple genome-scale assays, MultiAssayExperiment containers are used. See Figure \ref{fig:masc} for the layout.

\begin{figure}
\includegraphics[width=0.8\linewidth,]{MAEschema} \caption{MultiAssayExperiment schematic.}\label{fig:masc}
\end{figure}

Further details on these data structures will be provided in section \ref{class}.

\subsection{Downloading and caching cancer genomics data and annotations}\label{cache}}

Downloading and managing data from various online resources
can be excessively time consuming. Bioconductor encourages data caching for
increased efficiency and reproducibility. The caching data methods
employed in Bioconductor
allow analysis code to
concisely refer to data resources as needed, with minimal attention to how
data are stored, retrieved or transformed.
It allows for easy management and reuse of data that are on remote
servers or in cloud, storing source
location and providing information for data updates. The BiocFileCache
Bioconductor package handles data management from within R.

BiocFileCache is a general-use caching system but Bioconductor also provides
``Hubs'', AnnotationHub and ExperimentHub, to help distributed annotation or
experimental data hosted externally. Both AnnotationHub and ExperimentHub use
BiocFileCache to handle download and caching of data.

AnnotationHub provides a centralized repository of diverse genomic annotations,
facilitating easy access and integration into analyses. Researchers can
seamlessly retrieve information such as genomic features, functional
annotations, and variant data, streamlining the annotation process for their
analyses.

ExperimentHub extends this concept to experimental data. It serves as a
centralized hub for storing and sharing curated experiment-level datasets,
allowing researchers to access a wide range of experimental designs and
conditions. This cloud-based infrastructure enhances collaboration and promotes
the reproducibility of analyses across different laboratories.

The curatedTCGAData package provides some resources through
ExperimentHub, as do many other self-identified ``CancerData'' resources. Once the
ExperimentHub is loaded, it can be queried for terms of interest.

\begin{shaded}
\begin{verbatim}
library(ExperimentHub)
eh = ExperimentHub()
query(eh, "CancerData")
## ExperimentHub with 1742 records
## # snapshotDate(): 2024-04-29
## # $dataprovider: Eli and Edythe L. Broad Institute of Harvard and MIT, GEO, ...
## # $species: Homo sapiens, Mus musculus, NA
## # $rdataclass: SummarizedExperiment, RaggedExperiment, matrix, list, DFrame,...
## # additional mcols(): taxonomyid, genome, description,
## #   coordinate_1_based, maintainer, rdatadateadded, preparerclass, tags,
## #   rdatapath, sourceurl, sourcetype 
## # retrieve records with, e.g., 'object[["EH558"]]' 
## 
##            title                                
##   EH558  | ACC_CNASNP-20160128                  
##   EH559  | ACC_CNVSNP-20160128                  
##   EH560  | ACC_colData-20160128                 
##   EH561  | ACC_GISTIC_AllByGene-20160128        
##   EH562  | ACC_GISTIC_ThresholdedByGene-20160128
##   ...      ...                                  
##   EH8533 | tcga_transcript_counts               
##   EH8534 | target_rhabdoid_wgbs_hg19            
##   EH8567 | xenium_hs_breast_addon               
##   EH9482 | Capper_example_betas.rda             
##   EH9483 | GIMiCC_Library.rda    
\end{verbatim}
\end{shaded}

%\texttt{\{r useeh\} \textless{}!-\/- , fig.cap="ExperimentHub attachment, retrieval, query, and response when seeking cancer-related data.", message=FALSE\} -\/-\textgreater{} library(ExperimentHub) eh \textless{}- ExperimentHub() query(eh, "CancerData")}

Multiple terms can be used to narrow results before choosing a download.

\begin{shaded}
\begin{verbatim}
query(eh, c("CancerData", "esophageal"))
## ExperimentHub with 2 records}
## snapshotDate(): 2023-10-24}
## $dataprovider: University of California San Francisco}
## $species: Homo sapiens}
## $rdataclass: RangedSummarizedExperiment, data.frame}
## additional mcols(): taxonomyid, genome, description,}
##   coordinate_1_based, maintainer, rdatadateadded, preparerclass, tags,}
##   rdatapath, sourceurl, sourcetype }
## retrieve records with, e.g., object[["EH8527"]]
##            title                           
##   EH8527 | cao_esophageal_wgbs_hg19        
##   EH8530 | cao_esophageal_transcript_counts
\end{verbatim}
\end{shaded}

Similarly AnnotationHub files can be downloaded for annotating data. For example,
the ensembl 110 release of gene and protein annotations are obtained with the
following:

\begin{shaded}
\begin{verbatim}
library(AnnotationHub)
ah = AnnotationHub()
query(ah, c("ensembl", "110", "Homo sapiens"))
#snapshotDate(): 2024-04-29
#AnnotationHub with 1 record
## snapshotDate(): 2024-04-29
## names(): AH113665
## $dataprovider: Ensembl
## $species: Homo sapiens
## $rdataclass: EnsDb
## $rdatadateadded: 2023-04-25
## $title: Ensembl 110 EnsDb for Homo sapiens
## $description: Gene and protein annotations for Homo sapiens based on Ensem...
## $taxonomyid: 9606
## $genome: GRCh38
## $sourcetype: ensembl
## $sourceurl: http://www.ensembl.org
## $sourcesize: NA
## $tags: c("110", "Annotation", "AnnotationHubSoftware", "Coverage",
##   "DataImport", "EnsDb", "Ensembl", "Gene", "Protein", "Sequencing",
##   "Transcript") 
## retrieve record with 'object[["AH113665"]]' 
\end{verbatim}
\end{shaded}


\section{Exploring institutionally curated cancer genomics data}\label{exploring-institutionally-curated-cancer-genomics-data}}


\subsection{The Cancer Genome Atlas}\label{the-cancer-genome-atlas}}

An overview of Bioconductor's resource for the Cancer
Genome Atlas (TCGA) is easy to obtain, with the
curatedTCGAData package.

\begin{Shaded}
\begin{verbatim}
library(curatedTCGAData)
tcgatab = curatedTCGAData(version="2.1.1")
\end{verbatim}
\end{Shaded}
%\begin{Shaded}
%\begin{Highlighting}[]
%\KeywordTok{library}\NormalTok{(curatedTCGAData)}
%\NormalTok{tcgatab =}\StringTok{ }\KeywordTok{curatedTCGAData}\NormalTok{(}\DataTypeTok{version=}\StringTok{"2.1.1"}\NormalTok{)}
%\end{Highlighting}
%\end{Shaded}

Records obtained for adrenocortical carcinoma (code ACC) are in Table \ref{tab:tab-lktab}.

\begin{table}

\caption{\label{tab:tab-lktab}Records returned by curatedTCGAData::curatedTCGAData(), filtered to those pertaining to adrenocortical carcinoma.}
\centering
\begin{tabular}[t]{lllll}
\toprule
  & ah\_id & title & file\_size & rdataclass\\
\midrule
1 & EH4737 & ACC\_CNASNP-20160128 & 0.8 Mb & RaggedExperiment\\
2 & EH4738 & ACC\_CNVSNP-20160128 & 0.2 Mb & RaggedExperiment\\
3 & EH4740 & ACC\_GISTIC\_AllByGene-20160128 & 0.2 Mb & SummarizedExperiment\\
4 & EH4741 & ACC\_GISTIC\_Peaks-20160128 & 0 Mb & RangedSummarizedExperiment\\
5 & EH4742 & ACC\_GISTIC\_ThresholdedByGene-20160128 & 0.2 Mb & SummarizedExperiment\\
\addlinespace
6 & EH4744 & ACC\_Methylation-20160128\_assays & 239.2 Mb & SummarizedExperiment\\
7 & EH4745 & ACC\_Methylation-20160128\_se & 6 Mb & RaggedExperiment\\
8 & EH4747 & ACC\_Mutation-20160128 & 0.7 Mb & SummarizedExperiment\\
9 & EH4748 & ACC\_RNASeq2Gene-20160128 & 2.7 Mb & SummarizedExperiment\\
10 & EH4750 & ACC\_RPPAArray-20160128 & 0.1 Mb & SummarizedExperiment\\
\addlinespace
414 & EH8118 & ACC\_miRNASeqGene-20160128 & 0.2 Mb & SummarizedExperiment\\
415 & EH8119 & ACC\_RNASeq2GeneNorm-20160128 & 5.4 Mb & SummarizedExperiment\\
\bottomrule
\end{tabular}
\end{table}

Various conventions are in play in this table. The ``title'' field is
of primary concern. The title string can be decomposed into
substrings with interpretation
\texttt{{[}tumorcode{]}\_{[}assay{]}-{[}date{]}\_{[}optional codes{]}}. The column \texttt{ah\_id} will be
explained in section \ref{hubs}, and entries in column
\texttt{rdataclass} will be discussed in section \ref{class} below.


\subsubsection{Tumor code resolution}\label{tumor-code-resolution}}

There are 33 different tumor types available in TCGA. The
decoding of tumor codes for the first ten in alphabetical order is
provided in Table \ref{tab:tab-deco}.

\begin{table}

\caption{\label{tab:tab-deco}Decoding TCGA tumor code abbreviations.}
\centering
\begin{tabular}[t]{ll}
\toprule
Code & Tumor.Type\\
\midrule
ACC & Adrenocortical Carcinoma\\
BLCA & Bladder Urothelial Carcinoma\\
BRCA & Breast Invasive Carcinoma\\
CESC & Cervical Squamous Cell Carcinoma and Endocervical Adenocarcinoma\\
CHOL & Cholangiocarcinoma\\
\addlinespace
CNTL & Controls\\
COAD & Colon Adenocarcinoma\\
DLBC & Lymphoid Neoplasm Diffuse Large B-cell Lymphoma\\
ESCA & Esophageal Carcinoma\\
FPPP & FFPE Pilot Phase II\\
\addlinespace
GBM & Glioblastoma Multiforme\\
HNSC & Head and Neck Squamous Cell Carcinoma\\
KICH & Kidney Chromophobe\\
KIRC & Kidney Renal Clear Cell Carcinoma\\
KIRP & Kidney Renal Papillary Cell Carcinoma\\
\addlinespace
LAML & Acute Myeloid Leukemia\\
LCML & Chronic Myelogenous Leukemia\\
LGG & Brain Lower Grade Glioma\\
LIHC & Liver Hepatocellular Carcinoma\\
LUAD & Lung Adenocarcinoma\\
\addlinespace
LUSC & Lung Squamous Cell Carcinoma\\
MESO & Mesothelioma\\
MISC & Miscellaneous\\
OV & Ovarian Serous Cystadenocarcinoma\\
PAAD & Pancreatic Adenocarcinoma\\
\addlinespace
PCPG & Pheochromocytoma and Paraganglioma\\
PRAD & Prostate Adenocarcinoma\\
READ & Rectum Adenocarcinoma\\
SARC & Sarcoma\\
SKCM & Skin Cutaneous Melanoma\\
\addlinespace
STAD & Stomach Adenocarcinoma\\
TGCT & Testicular Germ Cell Tumors\\
THCA & Thyroid Carcinoma\\
THYM & Thymoma\\
UCEC & Uterine Corpus Endometrial Carcinoma\\
\addlinespace
UCS & Uterine Carcinosarcoma\\
UVM & Uveal Melanoma\\
\bottomrule
\end{tabular}
\end{table}


\subsubsection{Assay codes and counts}\label{assay-codes-and-counts}}

Assays performed on tumors vary across tumor types. For assay
types shared between
breast cancer, glioblastoma, and lung adenocarcinoma (code LUAD),
the numbers of tumor and normal samples available in curatedTCGAData
are provided in Table \ref{tab:tab-doassc}.

\begin{table}

\caption{\label{tab:tab-doassc}Numbers of assays available in TCGA on tumor and normal samples,
for breast cancer, glioblastoma, and lung adenocarcinoma.}
\centering
\begin{tabular}[t]{lrrrrrr}
\toprule
  & BRCA & BRCAnormal & GBM & GBMnormal & LUAD & LUADnormal\\
\midrule
CNASNP & 1089 & 1120 & 577 & 527 & 516 & 579\\
CNVSNP & 1080 & 1119 & 577 & 527 & 516 & 579\\
GISTIC\_AllByGene & 1080 & 0 & 577 & 0 & 516 & 0\\
GISTIC\_Peaks & 1080 & 0 & 577 & 0 & 516 & 0\\
GISTIC\_ThresholdedByGene & 1080 & 0 & 577 & 0 & 516 & 0\\
\addlinespace
Mutation & 988 & 5 & 283 & 7 & 230 & 0\\
RNASeq2Gene & 1093 & 119 & 153 & 13 & 515 & 61\\
RPPAArray & 887 & 50 & 233 & 11 & 365 & 0\\
RNASeq2GeneNorm & 1093 & 119 & 153 & 13 & 515 & 61\\
Methylation\_methyl27 & 314 & 29 & 285 & 0 & 65 & 24\\
\addlinespace
Methylation\_methyl450 & 783 & 102 & 140 & 14 & 458 & 34\\
\bottomrule
\end{tabular}
\end{table}


\subsubsection{An example dataset for RNA-seq from glioblastoma multiforme}\label{an-example-dataset-for-rna-seq-from-glioblastoma-multiforme}}

We obtain normalized RNA-seq data on primary tumor samples for GBM with

%\begin{Shaded}
%\begin{Highlighting}[]
%\NormalTok{gbrna =}\StringTok{ }\KeywordTok{TCGAprimaryTumors}\NormalTok{(}\KeywordTok{curatedTCGAData}\NormalTok{(}\StringTok{"GBM"}\NormalTok{, }
%    \StringTok{"RNASeq2GeneNorm"}\NormalTok{, }\DataTypeTok{dry.run=}\OtherTok{FALSE}\NormalTok{, }\DataTypeTok{version=}\StringTok{"2.1.1"}\NormalTok{))}
%\NormalTok{gbrna}
%\CommentTok{\#\# A MultiAssayExperiment object of 1 listed}
%\CommentTok{\#\#  experiment with a user{-}defined name and respective class.}
%\CommentTok{\#\#  Containing an ExperimentList class object of length 1:}
%\CommentTok{\#\#  [1] GBM\_RNASeq2GeneNorm{-}20160128: SummarizedExperiment with 18199 rows and 153 columns}
%\CommentTok{\#\# Functionality:}
%\CommentTok{\#\#  experiments() {-} obtain the ExperimentList instance}
%\CommentTok{\#\#  colData() {-} the primary/phenotype DataFrame}
%\CommentTok{\#\#  sampleMap() {-} the sample coordination DataFrame}
%\CommentTok{\#\#  \textasciigrave{}$\textasciigrave{}, \textasciigrave{}[\textasciigrave{}, \textasciigrave{}[[\textasciigrave{} {-} extract colData columns, subset, or experiment}
%\CommentTok{\#\#  *Format() {-} convert into a long or wide DataFrame}
%\CommentTok{\#\#  assays() {-} convert ExperimentList to a SimpleList of matrices}
%\CommentTok{\#\#  exportClass() {-} save data to flat files}
%\end{Highlighting}
%\end{Shaded}

\begin{Shaded}
\begin{verbatim}
gbrna = TCGAprimaryTumors(curatedTCGAData("GBM",
"RNASeq2GeneNorm", dry.run=FALSE, version="2.1.1"))
gbrna
## A MultiAssayExperiment object of 1 listed
##experiment with a user-defined name and respective class.
##Containing an ExperimentList class object of length 1:
[1] GBM_RNASeq2GeneNorm-20160128: SummarizedExperiment with 18199 rows and 153 columns
##
## Functionality:
##experiments() - obtain the ExperimentList instance
##colData() - the primary/phenotype DataFrame
##sampleMap() - the sample coordination DataFrame
##`$`, `[`, `[[` - extract colData columns, subset, or experiment
####*Format() - convert into a long or wide DataFrame
assays() - convert ExperimentList to a SimpleList of matrices
##exportClass() - save data to flat files
\end{verbatim}
\end{Shaded}

R functions defined in Bioconductor packages can operate on the variable \texttt{gbrna} to
retrieve information of interest. Details on the underlying data structure
are given in section \ref{class} below. For most assay types, we think of the quantitative
assay
information as tabular in nature, with table rows corresponding to genomic
features such as genes, and table columns corresponding to samples.

Information on GBM samples employs the \texttt{colData} function.

%\begin{Shaded}
%\begin{Highlighting}[]
%\KeywordTok{dim}\NormalTok{(}\KeywordTok{colData}\NormalTok{(gbrna))}
%\CommentTok{\#\# [1]  153 4380}
%\end{Highlighting}
%\end{Shaded}


\begin{verbatim}
dim(colData(gbrna))
## [1]
153 4380
\end{verbatim}

For sample level information obtained \texttt{colData}, we think of rows
as samples, and columns as sample attributes.


\subsubsection{Clinical and phenotypic data}
\label{clinical-and-phenotypic-data}}

TCGA datasets are generally provided as combinations of
results for tumor tissue and normal tissue. The determination
of a record's sample type is encoded in the sample ``barcode''.
Decoding of sample barcodes is described at the \href{https://docs.gdc.cancer.gov/Encyclopedia/pages/TCGA_Barcode/}{Genomic Data Commons Encyclopedia} with specific interpretation of sample types listed \href{https://gdc.cancer.gov/resources-tcga-users/tcga-code-tables/sample-type-codes}{separately}. The TCGAutils package provides utilities for extracting
data on primary tumor samples, excluding samples that may have been taken on
normal tissue or metastases.

Clinical and phenotypic data on all TCGA samples are voluminous. For example,
there are 2684 fields of sample level data for BRCA
samples, and 4380 fields for GBM samples. Many of these
fields are meaningfully populated for only a very small minority of samples.
To see this for GBM:

%\begin{Shaded}
%\begin{Highlighting}[]
%\KeywordTok{mean}\NormalTok{(}\KeywordTok{sapply}\NormalTok{(}\KeywordTok{colData}\NormalTok{(gbrna), }\ControlFlowTok{function}\NormalTok{(x) }\KeywordTok{mean}\NormalTok{(}\KeywordTok{is.na}\NormalTok{(x))}\OperatorTok{\textgreater{}}\NormalTok{.}\DecValTok{90}\NormalTok{))}
%\CommentTok{\#\# [1] 0.8091324}
%\end{Highlighting}
%\end{Shaded}

\begin{verbatim}
mean(sapply(colData(gbrna), function(x) mean(is.na(x))>.90))
## [1] 0.8091324
\end{verbatim}

In words, for 81\% of clinical data fields in TCGA GBM data,
at least 90\% of entries are missing.

Nevertheless, with careful inspection of fields and contents,
nearly complete clinical data can be extracted and combined with molecular
and genetic assay data with modest effort.

The following code chunk illustrates a very crude
approach to comparing survival profiles for BRCA, GBM, and LUAD
donors. The result is in Figure \ref{fig:dothesurv}.


\begin{verbatim}
# obtain mutation data for BRCA, GBM, LUAD; could use any or all assay types
brmut = curatedTCGAData("BRCA", "Mutation", version = "2.1.1", dry.run = FALSE)
gbmut = curatedTCGAData("GBM", "Mutation", version = "2.1.1", dry.run = FALSE)
lumut = curatedTCGAData("LUAD", "Mutation", version = "2.1.1", dry.run = FALSE)
# extract survival times
library(survival)
getSurv = function(mae) {
days_on = with(colData(mae), ifelse(is.na(days_to_last_followup),
days_to_death, days_to_last_followup))
8Bioconductor’s Computational Ecosystem for Genomic Data Science in Cancer
Surv(days_on, colData(mae)$vital_status)
}
ss = lapply(list(brmut, gbmut, lumut), getSurv)
codes = c("BRCA", "GBM", "LUAD")
type = factor(rep(codes, sapply(ss,length)))
allsurv = do.call(c, ss)
library(GGally)
ggsurv(survfit(allsurv~type))
\end{verbatim}

\begin{figure}
\includegraphics[width=0.8\linewidth,]{bioccb_files/figure-latex/dothesurv-1} \caption{Survival profile extraction from three MultiAssayExperiments produced with curatedTCGAData calls.}\label{fig:dothesurv}
\end{figure}

At this point, survival times within tumor type can be stratified by any
features of the mutation profiles of individual samples.
The ``RaggedExperiment'' class is employed to test each BRCA sample for
presence of any mutation in the gene TTN. See Figure \ref{fig:strat}.

\begin{verbatim}
bprim = TCGAprimaryTumors(brmut)
## harmonizing input:
##
removing 5 sampleMap rows with 'colname' not in colnames of experiments
mutsyms = assay(experiments(bprim)[[1]], "Hugo_Symbol")
cn = rownames(colData(bprim)) # short
cna = colnames(mutsyms) # long
cnas = substr(cna, 1, 12)
hasTTNmut = apply(assay(experiments(TCGAprimaryTumors(brmut))[[1]], "Hugo_Symbol"),
2, function(x) length(which(x=="TTN"))>0)
## harmonizing input:
##
removing 5 sampleMap rows with 'colname' not in colnames of experiments
names(hasTTNmut) = cnas
bsurv = getSurv(TCGAprimaryTumors(brmut))
## harmonizing input:
##
removing 5 sampleMap rows with 'colname' not in colnames of experiments
hasTTNmut = hasTTNmut[cn] # match mutation records to surv times
ggsurv(survfit(bsurv~hasTTNmut))
\end{verbatim}


\begin{figure}
\includegraphics[width=1\linewidth,]{bioccb_files/figure-latex/strat-1} \caption{Survival distributions for donors of breast tumors in TCGA, stratified by presence or absence of mutation in gene TTN.}\label{fig:strat}
\end{figure}

Similar manipulations permit exploration of relationships between
any molecular assay outcomes and any clinical data collected in TCGA.


\subsection{cBioPortal}\label{cbioportal}}

The \href{https://www.cbioportal.org/}{cBioPortal} user guide
defines the goal of the portal to be reducing ``the barriers between complex
genomic data and cancer researchers by providing rapid, intuitive, and high-quality
access to molecular profiles and clinical attributes from large-scale cancer genomics projects, and
therefore to empower researchers to translate these rich data sets into biologic insights and clinical applications.''

Bioconductor's cBioPortalData package simplifies access to over 300 genomic studies of
diverse cancers in cBioPortal. The main unit of data access is the publication. The
\texttt{cBioPortal} function mediates a connection between an R session and the
cBioPortal API. \texttt{getStudies} returns a tibble with metadata on
all studies.

\begin{Shaded}
\begin{verbatim}
library(cBioPortalData)
cbio = cBioPortal()
allst = getStudies(cbio)
dim(allst)
## [1] 397
13
\end{verbatim}
\end{Shaded}

A pruned selection of records from the cBioPortal
studies table is given in Table \ref{tab:tab-cball}.

\begin{table}
\caption{\label{tab:tab-cball}Excerpts from four fields on selected records in the cBioPortal getStudies output.}\\
\begin{tabular}{p{5cm}p{5cm}l}
name & description & studyId \\ \hline
Adenoid Cystic Carcinoma of the Breast & Whole exome sequencing of 12 breast AdCCs. & acbc\_mskcc\_2015 \\
Adenoid Cystic Carcinoma & Whole-exome or whole-genome sequencing analysis of 60 ACC tumor/normal pairs & acyc\_mskcc\_2013 \\
Adenoid Cystic Carcinoma & Targeted Sequencing of 28 metastatic Adenoid Cystic Carcinoma samples. & acyc\_fmi\_2014 \\
Adenoid Cystic Carcinoma & Whole-genome or whole-exome sequencing of 25 adenoid cystic carcinoma tumor/normal pairs. & acyc\_jhu\_2016 \\
Adenoid Cystic Carcinoma & WGS of 21 salivary ACCs and targeted molecular analyses of a validation set (81 patients). & acyc\_mda\_2015 \\
Adenoid Cystic Carcinoma & Whole-genome/exome sequencing of 10 ACC PDX models. & acyc\_mgh\_2016 \\
Adenoid Cystic Carcinoma & Whole exome sequencing of 24 ACCs. & acyc\_sanger\_2013 \\
Adenoid Cystic Carcinoma Project & Multi-Institute Cohort of 1045 Adenoid Cystic Carcinoma patients. & acc\_2019 \\
Basal Cell Carcinoma & Whole-exome sequencing of 126 basal cell carcinoma tumor/normal pairs; targeted sequencing of 163 sporadic samples (40 tumor/normal pairs) and 4 Gorlin symdrome basal cell carcinomas. & bcc\_unige\_2016 \\
\end{tabular}
\end{table}

%\begin{table}[lll]%{>{\raggedright\arraybackslash}p{12em}>{\raggedright\arraybackslash}p{15em}l}
%\caption{\label{tab:tab-cball}Excerpts from four fields on selected records in the cBioPortal getStudies output.}\\
%\toprule
%name & description & studyId\\
%\midrule
%Adenoid Cystic Carcinoma of the Breast & Whole exome sequencing of 12 breast AdCCs. & acbc\_mskcc\_2015\\
%Adenoid Cystic Carcinoma & Whole-exome or whole-genome sequencing analysis of 60 ACC tumor/normal pairs & acyc\_mskcc\_2013\\
%Adenoid Cystic Carcinoma & Targeted Sequencing of 28 metastatic Adenoid Cystic Carcinoma samples. & acyc\_fmi\_2014\\
%Adenoid Cystic Carcinoma & Whole-genome or whole-exome sequencing of 25 adenoid cystic carcinoma tumor/normal pairs. & acyc\_jhu\_2016\\
%Adenoid Cystic Carcinoma & WGS of 21 salivary ACCs and targeted molecular analyses of a validation set (81 patients). & acyc\_mda\_2015\\
%\addlinespace
%Adenoid Cystic Carcinoma & Whole-genome/exome sequencing of 10 ACC PDX models. & acyc\_mgh\_2016\\
%Adenoid Cystic Carcinoma & Whole exome sequencing of 24 ACCs. & acyc\_sanger\_2013\\
%Adenoid Cystic Carcinoma Project & Multi-Institute Cohort of 1045 Adenoid Cystic Carcinoma patients. & acc\_2019\\
%Basal Cell Carcinoma & Whole-exome sequencing of 126 basal cell carcinoma tumor/normal pairs; targeted sequencing of 163 sporadic samples (40 tumor/normal pairs) and 4 Gorlin symdrome basal cell carcinomas. & bcc\_unige\_2016\\
%\bottomrule
%\end{table}

To explore copy number alteration data from a study on angiosarcoma,
we find the associated studyId field in \texttt{allst} and use the \texttt{cBioDataPack} function
to retrieve a MultiAssayExperiment:

\begin{verbatim}
ann = "angs_project_painter_2018"
ang = cBioDataPack(ann)
## Warning in .find_with_xfix(df_colnames, get(paste0(fix, 1)), get(paste0(fix, :
## Multiple prefixes found, using keyword 'region' or taking first one
## Warning in .find_with_xfix(df_colnames, get(paste0(fix, 1)), get(paste0(fix, :
## Multiple prefixes found, using keyword 'region' or taking first one
ang
## A MultiAssayExperiment object of 3 listed
##experiments with user-defined names and respective classes.
####Containing an ExperimentList class object of length 3:
[1] cna_hg19.seg: RaggedExperiment with 27835 rows and 48 columns
##[2] cna: SummarizedExperiment with 23109 rows and 48 columns
##[3] mutations: RaggedExperiment with 24058 rows and 48 columns
## Functionality:
##experiments() - obtain the ExperimentList instance
##colData() - the primary/phenotype DataFrame
##sampleMap() - the sample coordination DataFrame
##`$`, `[`, `[[` - extract colData columns, subset, or experiment
####*Format() - convert into a long or wide DataFrame
assays() - convert ExperimentList to a SimpleList of matrices
##exportClass() - save data to flat files
\end{verbatim}

The copy number alteration outcomes are in the
\texttt{assay} component of the experiment.

\begin{Shaded}
\begin{verbatim}
seg = experiments(ang)[[1]]
colnames(seg) = sapply(strsplit(colnames(seg), "-"), "[", 5)
assay(seg)[1:4,1:4]
##
##                   DAE1F DACME DADBW DAD34
## 1:12227-955755       71    NA    NA    NA
## 1:957844-1139868     62    NA    NA    NA
## 1:1140874-1471177   167    NA    NA    NA
## 1:1475170-1855370   113    NA    NA    NA
\end{verbatim}
\end{Shaded}

The rownames component of this matrix can be transformed to
a GenomicRanges instance for concise manipulation.

%\begin{Shaded}
%\begin{Highlighting}[]
%\KeywordTok{library}\NormalTok{(GenomicRanges)}
%\CommentTok{\#\# 0/0 packages newly attached/loaded, see sessionInfo() for details.}
%\KeywordTok{library}\NormalTok{(ggplot2)}
%\CommentTok{\#\# 0/0 packages newly attached/loaded, see sessionInfo() for details.}
%\NormalTok{allalt =}\StringTok{ }\KeywordTok{GRanges}\NormalTok{(}\KeywordTok{rownames}\NormalTok{(}\KeywordTok{assay}\NormalTok{(seg)))}
%\NormalTok{allalt}
%\CommentTok{\#\# GRanges object with 27835 ranges and 0 metadata columns:}
%\CommentTok{\#\#           seqnames            ranges strand}
%\CommentTok{\#\#              \textless{}Rle\textgreater{}         \textless{}IRanges\textgreater{}  \textless{}Rle\textgreater{}}
%\CommentTok{\#\#       [1]        1      12227{-}955755      *}
%\CommentTok{\#\#       [2]        1    957844{-}1139868      *}
%\CommentTok{\#\#       [3]        1   1140874{-}1471177      *}
%\CommentTok{\#\#       [4]        1   1475170{-}1855370      *}
%\CommentTok{\#\#       [5]        1  1857786{-}17257894      *}
%\CommentTok{\#\#       ...      ...               ...    ...}
%\CommentTok{\#\#   [27831]       20     68410{-}1559342      *}
%\CommentTok{\#\#   [27832]       20   1585705{-}1592359      *}
%\CommentTok{\#\#   [27833]       20  1616247{-}62904955      *}
%\CommentTok{\#\#   [27834]       21  9907492{-}48084286      *}
%\CommentTok{\#\#   [27835]       22 16157938{-}51237572      *}
%\CommentTok{\#\#   {-}{-}{-}{-}{-}{-}{-}}
%\CommentTok{\#\#   seqinfo: 22 sequences from an unspecified genome; no seqlengths}
%\end{Highlighting}
%\end{Shaded}

We'll focus on chromosome 17, where TP53 is found. Regions
of genomic alteration are summarized to their midpoints.

%\begin{Shaded}
%\begin{Highlighting}[]
%\NormalTok{g17 =}\StringTok{ }\NormalTok{allalt[}\KeywordTok{seqnames}\NormalTok{(allalt)}\OperatorTok{==}\StringTok{"17"}\NormalTok{]}
%\NormalTok{df17 =}\StringTok{ }\KeywordTok{as}\NormalTok{(g17, }\StringTok{"data.frame"}\NormalTok{)        }\CommentTok{\# for ggplot2}
%\NormalTok{df17}\OperatorTok{$}\NormalTok{mid =}\StringTok{ }\FloatTok{.5}\OperatorTok{*}\NormalTok{(df17}\OperatorTok{$}\NormalTok{start}\OperatorTok{+}\NormalTok{df17}\OperatorTok{$}\NormalTok{end) }\CommentTok{\# midpoint only}
%\KeywordTok{ggplot}\NormalTok{(df17, }\KeywordTok{aes}\NormalTok{(}\DataTypeTok{x=}\NormalTok{mid)) }\OperatorTok{+}\StringTok{ }\KeywordTok{geom\_density}\NormalTok{(}\DataTypeTok{bw=}\NormalTok{.}\DecValTok{2}\NormalTok{) }\OperatorTok{+}\StringTok{ }\KeywordTok{xlab}\NormalTok{(}\StringTok{"chr 17 bp"}\NormalTok{)}
%\end{Highlighting}
%\end{Shaded}

\begin{figure}
\includegraphics[width=1\linewidth,]{bioccb_files/figure-latex/mkden-1} \caption{Density of recurrent genomic alterations on chromosome 17 for 48 angiosarcoma patients.}\label{fig:mkden}
\end{figure}

This display shows a strong peak in the vicinity of 7.5 Mb on chromosome 17, near TP53.




\section{Analytical workflows}\label{analytical-workflows}


\subsection{Overview}\label{overview}

Table \ref{tab:tab-wflow} presents an informal topical
labeling for Bioconductor software packages with
cancer mentioned in the Description field of package
metadata.

\begin{table}
\caption{\label{tab:tab-wflow}Topical organization of packages with cancer applications.}
\begin{tabular}{l{4cm}p{6cm}}
\toprule
topic & packages\\
\midrule
Ancestry & RAIDS\\
Biomarkers & INDEED, iPath, RLassoCox\\
ceRNA & GDCRNATools\\
Clonal Evolution & CIMICE, LACE, OncoSimulR, TRONCO, CancerInSilico, cellscape\\
CNV & oncoscanR, SCOPE, ZygosityPredictor\\
\addlinespace
DrugSensitivity & DepInfeR, octad, PharmacoGx, rcellminer\\
Epigenetics & MethylMix, AMARETTO, COCOA, methylclock, missMethyl\\
HotSpots/Drivers/signatures & compSPOT, MoonlightR, Moonlight2R, \\
 & DriverNet, genefu, mastR, pathifier, RESOLVE, macat, \\
 & SigCheck, signeR, signifinder, supersigs, decompTumor2Sig, YAPSA\\
ImmuneModulation & easier\\
IsoformSwitching & IsoformSwitchAnalyzeR\\
\addlinespace
Literature mining & OncoScore\\
ncRNA & NoRCE\\
Radiomics & RadioGx\\
RecurrentFusion & copa, oppar\\
Spatial & SpatialDecon\\
\addlinespace
SpecificCancers & consensusOV, PDATK, STROMA4\\
Splicing & OutSplice, psichomics\\
Subtyping & SCFA\\
\end{tabular}
\bottomrule
\end{table}

The vignettes of each of these packages provide background and
illustration of their roles in cancer genomics.

\subsection{Packages supporting epigenomic analysis}\label{packages-supporting-epigenomic-analysis}

Bioconductor also provides a diverse array of packages for analysis of epigenome
data. Cancer is often studied under a developmental lens, so increasingly, studies
are measuring cell states using epigenomic methods. Epigenomics is the study of
chemical modifications and chromosomal conformations of DNA in a nucleus; in cancer
epigenomics, we study how the cancer epigenome differs among cancers and how
these relate to healthy epigenomes. As of 2023, Bioconductor includes 89 packages
under \emph{Epigenetics} and 93 packages tagged under \emph{FunctionalGenomics}, including dozens of tools
for analyzing a variety of epigenome assays, such as ATAC-seq, ChIP-seq, or
bisulfite-seq. Among these are also tools that handle more general analysis, such
as genomic region set enrichment.

First, for ATAC-seq data, bioconductor packages include general-purpose pipelines, including scPipe
\cite{Tian2018}. %(Tian et al. \protect\hyperlink{ref-Tian2018}{2018})
and esATAC \cite{Wei2018} %(Wei et al. \protect\hyperlink{ref-Wei2018}{2018}), 
which start from FASTQ files and produce feature count
matrices. Alternatively, many practitioners elect to do general-purpose pipeline processing outside of
R, and then bring the processed data into R for statistical analysis,
visualization, and quality control. In this approach, ATACseqQC
provides
a variety of QC plots specific to ATAC-seq data \cite{Ou2018}.% (Ou et al. \protect\hyperlink{ref-Ou2018}{2018}).

For DNA methylation, many popular packages have been developed to help with
all stages of a DNA methylation analysis. These include minfi 
\cite{Aryee2014}
which specializes in methylation array analysis, biseq and bsseq \cite{Hansen2012}  %(Hansen, Irizarry, and Wu \protect\hyperlink{ref-Hansen2012}{2012})
which provide fundamental infrastructure for sequencing-based assays, and RnBeads
\cite{Mueller2019},
%(Müller et al. \protect\hyperlink{ref-Mueller2019}{2019}), 
which provides a comprehensive general-purpose analysis of DNA
methylation cohorts from arrays or sequencing-based assays. Other packages provide more specialized
analysis approaches, such as MIRA \cite{Lawson2018}, %(Lawson et al. \protect\hyperlink{ref-Lawson2018}{2018}), 
which infers regulatory
activity of transcription factors using DNA methylation signals, %(Sheffield et al.~2018), FIXME not found
or ELMER, which uses DNA methylation and gene expression in large cancer
cohorts to infer transcription factor networks \cite{Silva2019}. % (Silva et al. \protect\hyperlink{ref-Silva2019}{2018}). 
EpiDISH infers
the proportions of cell-types present in a bulk sample on the basis
of DNA methylation data \cite{Zheng2018a}. %(Zheng et al. \protect\hyperlink{ref-Zheng2018a}{2018}).

%Another popular epigenome experiment is ChIP-seq, and Bioconductor delivers many packages in
%this area. 
DiffBind \cite{Stark2011} %(Stark and Brown \protect\hyperlink{ref-Stark2011}{2011}) is a popular approach for
facilitates differential binding analysis of ChIP-seq peak data.

%A variety of packages are also geared toward visualization of this type
%of data. 
GenomicDistributions \cite{Kupkova2022} % (Kupkova et al. \protect\hyperlink{ref-Kupkova2022}{2022}) 
provides a variety of plots for visualization
distributions of any type of genomic range data. The chromPlot package specializes
in plots across chromosomes.  Several packages deal with
unsupervised exploration of variation in epigenomic data. PathwayPCA, MOFA2 \cite{Argelaguet2020} %(Argelaguet et al. \protect\hyperlink{ref-Argelaguet2020}{2020})
and COCOA \cite{Lawson2020} %(Lawson et al. \protect\hyperlink{ref-Lawson2020}{2020}) 
can process any epigenomic signal data.
A variety of alternative approaches for enrichment analysis, which include LOLA \cite{Sheffield2016}, %(Sheffield and Bock \protect\hyperlink{ref-Sheffield2016}{2016}),
chipenrich, regionR \cite{Gel2015}, %(Gel et al. \protect\hyperlink{ref-Gel2015}{2015}), 
and FGNet \cite{Aibar2015}. %(Aibar et al. \protect\hyperlink{ref-Aibar2015}{2015}).
Annotation packages include ChIPpeakAnno \cite{Zhu2010} % (Zhu et al. \protect\hyperlink{ref-Zhu2010}{2010})
and annotatr \cite{Cavalcante2017}.%(Cavalcante and Sartor \protect\hyperlink{ref-Cavalcante2017}{2017}) are popular packages for annotating genomic
%ranges.


\subsection{Some details on prediction of responsiveness to immune checkpoint blockade}\label{some-details-on-prediction-of-responsiveness-to-immune-checkpoint-blockade}

The National Cancer Institute website on checkpoint inhibitors
in cancer immunotherapy (``Immune Checkpoint Inhibitors'' 
\cite{ICBnci}) %\protect\hyperlink{ref-ICBnci}{2022})
lists 12 different cancer types
amenable to treatment via immune checkpoint inhibition.
The ``easier'' package in Bioconductor
assembles multiple systems biology resources
to produce patient-specific
prediction of responsiveness to
immune checkpoint blockade (ICB) \cite{easierPap}. %, as described in Lapuente-Santana et al. (\protect\hyperlink{ref-easierPap}{2021}).

Figure \ref{fig:easfin} presents on overview of results of
immune response assessment in a cohort of patients with
bladder cancer \cite{Mariathasan2018}. %reported in Mariathasan et al. (\protect\hyperlink{ref-Mariathasan2018}{2018}).
Patient's bulk RNA-seq data are used to develop multiple
quantitative descriptors of the tumor microenvironment,
and scores for processes regarded as hallmarks of anti-cancer
immune responses.

\begin{figure}
\includegraphics[width=0.95\linewidth,]{easierFinal} \caption{Comparison of genomic features distinguishing patients non-responsive and responsive to immune checkpoint blockade.}\label{fig:easfin}
\end{figure}

This display encapsulates a) the capacity of measurements of
genomic elements to discriminate patients who respond
to ICB for bladder cancer (position of labeled
item on x axis), b) the direction of association of
element activity with immune response (shape of glyph) and c) the
relative magnitudes of weights (size of glyph) estimated for features in
initial model fitting.

The design of this package is noteworthy in its approach
to information hiding. Parameters estimated in machine
learning of tissue-specific relations between quantitative
descriptors of the tumor microenvironment and hallmarks
of immune response are stored in ExperimentHub.

\begin{shaded}
\begin{verbatim}
library(easierData)
list_easierData()
##   eh_id                            title
##  EH6677   Mariathasan2018_PDL1_treatment
##  EH6678                       opt_models
##  EH6679                 opt_xtrain_stats
##  EH6680              TCGA_mean_pancancer
##  EH6681                TCGA_sd_pancancer
##  EH6682                 cor_scores_genes
##  EH6683               intercell networks
##  EH6684                lr_frequency_TCGA
##  EH6685                   group_lr_pairs
##  EH6686                  HGNC_annotation
##  EH6687           scores_signature_genes
\end{verbatim}
\end{shaded}

The structure of the stored model weights resource can be sketched by probing list elements.

\begin{shaded}
\begin{verbatim}
mw = eh[["EH6678"]]
## see ?easierData and browseVignettes('easierData') for documentation
## loading from cache
names(mw)   # TCGA tumor types
##  [1] "LUAD" "LUSC" "BLCA" "BRCA" "CESC" "CRC"  "GBM"  "HNSC" "KIRC"
## [10] "KIRP" "LIHC"   "OV" "PAAD" "PRAD" "SKCM" "STAD" "THCA" "UCEC"
## [19] "NSCLC"
names(mw[["LUAD"]]) # TME descriptors
## [1] "pathways" "immunecells" "tfs" "lrpairs" "ccpairs"
rownames(mw[["LUAD"]]$pathways$CYT) # predict cytolytic activity
##  [1] "(Intercept)" "Androgen" "EGFR" "Estrogen" "Hypoxia"
##  [6]    "JAK-STAT"     "MAPK" "NFkB" "p53"      "PI3K"
## [11]        "TNFa"    "Trail" "VEGF" "WNT"
\end{verbatim}
\end{shaded}


The vignette of the easier package steps through phases,
using these tumor-type-specific weights to compute patient-specific measures
of transcription factor activity or cell-cell interaction on the basis of bulk
RNA-seq (units are transcripts per million), and a patient-specific
measure of pathway activity using raw RNA-seq counts. These metrics
may be of interest in their own right for applications other than
establishing predictions of response to ICB.

Section \ref{app3} provides the names and versions of all packages
used to produce this analysis.



\subsection{Representing and visualizing spatial transcriptomics experiments}\label{representing-and-visualizing-spatial-transcriptomics-experiments}}

Spatial transcriptomics (ST) allows the quantification of RNA expression of large numbers of genes while preserving the spatial context of tissues and cells. This is important as cancer progression depends on a complex tumor microenvironment, and not just cell type composition, but also cell type spatial organization can be used to derive diagnostic or prognostic markers.

The Bioconductor project offers multiple approaches to handle and manipulate\\
spatial transcriptomics (ST) data.
The SpatialExperiment class (Righelli et al. \protect\hyperlink{ref-rig22}{2022}) is designed to be a lightweight,
technology-agnostic container. By inheriting from the
SingleCellExperiment class, it unlocks the use in ST data of
analysis packages designed for single-cell data, such as scater for exploration
and QC, and scran for normalization.
SpatialFeatureExperiment (Moses et al. \protect\hyperlink{ref-moses23}{2023}) extends SpatialExperiment to easily
reuse polygons and other spatial geometry features from geospatial CRAN
packages, such as sf. See also MoleculeExperiment (Couto et al. \protect\hyperlink{ref-Couto2023}{2023}) for a different
approach based on the data.table package.

In addition to data containers, Bioconductor provides a rich set of ST data.
The STexampleData and SFEData packages contain a collection of datasets from
different technologies and tissues.
The TENxVisiumData package provides a collection of 13 in-house 10X Genomics
Visium datasets from 23 samples across two organisms (human and mouse) and 13
tissues.
The MerfishData package contains two annotated samples assayed with the MERFISH
in-situ imaging protocol.

Finally, Bioconductor offers a growing collection of analysis methods tailored
for spot-based and in-situ ST data, including methods for visualization,
data exploration and quality control, spot deconvolution,
spatially-aware clustering, and identification of spatially-variable genes.

To show a simple example of an analysis workflow on spot-based data,
we explore a fresh frozen
Invasive Ductal Carcinoma breast tissue assayed
with the 10X Genomics Visium platform.
First, we use the ggspavis package for visualization.

\begin{Shaded}
\begin{Highlighting}[]
\KeywordTok{library}\NormalTok{(TENxVisiumData)}
\CommentTok{\#\# snapshotDate(): 2023{-}10{-}24}
\CommentTok{\#\# 3/1 packages newly attached/loaded, see sessionInfo() for details.}
\KeywordTok{library}\NormalTok{(SpatialExperiment)}
\CommentTok{\#\# 0/0 packages newly attached/loaded, see sessionInfo() for details.}
\KeywordTok{library}\NormalTok{(ggspavis)}
\CommentTok{\#\# 1/1 packages newly attached/loaded, see sessionInfo() for details.}
\NormalTok{hbc \textless{}{-}}\StringTok{ }\KeywordTok{HumanBreastCancerIDC}\NormalTok{()}
\CommentTok{\#\# see ?TENxVisiumData and browseVignettes(\textquotesingle{}TENxVisiumData\textquotesingle{}) for documentation}
\CommentTok{\#\# loading from cache}
\NormalTok{hbc \textless{}{-}}\StringTok{ }\NormalTok{hbc[,hbc}\OperatorTok{$}\NormalTok{sample\_id}\OperatorTok{==}\StringTok{"HumanBreastCancerIDC1"}\NormalTok{]}
\NormalTok{hbc}\OperatorTok{$}\NormalTok{in\_tissue \textless{}{-}}\StringTok{ }\OtherTok{TRUE}
\NormalTok{hbc \textless{}{-}}\StringTok{ }\KeywordTok{rotateImg}\NormalTok{(hbc, }\DataTypeTok{degrees=}\OperatorTok{{-}}\DecValTok{90}\NormalTok{)}
\KeywordTok{plotVisium}\NormalTok{(hbc, }\DataTypeTok{y\_reverse =} \OtherTok{FALSE}\NormalTok{)}
\end{Highlighting}
\end{Shaded}

\begin{figure}
\includegraphics[width=1\linewidth,]{bioccb_files/figure-latex/tenxvisium-1} \caption{Visualization of a Visium breast cancer sample}\label{fig:tenxvisium}
\end{figure}

To investigate the spatially variable genes the nnSVG package
implements a method for the detection of genes whose expression varies in the
tissue spatial domains by fitting nearest-neighbor Gaussian processes (Weber et al. \protect\hyperlink{ref-webr23}{2023}).

\begin{Shaded}
\begin{Highlighting}[]
\KeywordTok{library}\NormalTok{(scater)}
\CommentTok{\#\# 2/14 packages newly attached/loaded, see sessionInfo() for details.}
\KeywordTok{library}\NormalTok{(nnSVG)}
\CommentTok{\#\# 1/4 packages newly attached/loaded, see sessionInfo() for details.}
\KeywordTok{library}\NormalTok{(scran)}
\CommentTok{\#\# 1/9 packages newly attached/loaded, see sessionInfo() for details.}

\CommentTok{\#add quality metrics}
\NormalTok{is\_mito \textless{}{-}}\StringTok{ }\KeywordTok{grepl}\NormalTok{(}\StringTok{"(\^{}MT{-})|(\^{}mt{-})"}\NormalTok{, }\KeywordTok{rowData}\NormalTok{(hbc)}\OperatorTok{$}\NormalTok{symbol)}
\NormalTok{hbc \textless{}{-}}\StringTok{ }\KeywordTok{addPerCellQC}\NormalTok{(hbc, }\DataTypeTok{subsets =} \KeywordTok{list}\NormalTok{(}\DataTypeTok{mito =}\NormalTok{ is\_mito))}

\CommentTok{\#\# needed because the column name is hard coded in the nnSVG::filter\_genes}
\KeywordTok{rowData}\NormalTok{(hbc)}\OperatorTok{$}\NormalTok{gene\_name \textless{}{-}}\StringTok{ }\KeywordTok{rowData}\NormalTok{(hbc)}\OperatorTok{$}\NormalTok{symbol }

\CommentTok{\#\# filter and normalize gene expression}
\NormalTok{hbc \textless{}{-}}\StringTok{ }\KeywordTok{filter\_genes}\NormalTok{(hbc)}
\CommentTok{\#\# Gene filtering: removing mitochondrial genes}
\CommentTok{\#\# removed 13 mitochondrial genes}
\CommentTok{\#\# Gene filtering: retaining genes with at least 3 counts in at least 0.5\% (n = 19) of spatial locations}
\CommentTok{\#\# removed 26583 out of 36588 genes due to low expression}
\NormalTok{hbc \textless{}{-}}\StringTok{ }\KeywordTok{computeLibraryFactors}\NormalTok{(hbc)}
\NormalTok{hbc \textless{}{-}}\StringTok{ }\KeywordTok{logNormCounts}\NormalTok{(hbc)}

\CommentTok{\#\# select highly variable genes}
\NormalTok{hvgs \textless{}{-}}\StringTok{ }\KeywordTok{getTopHVGs}\NormalTok{(hbc, }\DataTypeTok{n=}\DecValTok{1000}\NormalTok{)}
\NormalTok{hbc \textless{}{-}}\StringTok{ }\NormalTok{hbc[hvgs,]}

\CommentTok{\#\# identify spatially variable genes}
\NormalTok{hbc \textless{}{-}}\StringTok{ }\KeywordTok{nnSVG}\NormalTok{(hbc, }\DataTypeTok{n\_threads=}\DecValTok{4}\NormalTok{)}

\CommentTok{\#\# post{-}processing}
\NormalTok{hbc \textless{}{-}}\StringTok{ }\NormalTok{hbc[}\KeywordTok{order}\NormalTok{(}\KeywordTok{rowData}\NormalTok{(hbc)}\OperatorTok{$}\NormalTok{rank),]}

\NormalTok{gnr1 \textless{}{-}}\StringTok{ }\KeywordTok{rowData}\NormalTok{(hbc)}\OperatorTok{$}\NormalTok{symbol[}\DecValTok{1}\NormalTok{]}
\KeywordTok{rownames}\NormalTok{(hbc) \textless{}{-}}\StringTok{ }\KeywordTok{rowData}\NormalTok{(hbc)}\OperatorTok{$}\NormalTok{symbol}
\end{Highlighting}
\end{Shaded}

By ranking the results of nnSVG, we are able to detect the most spatially
variable genes. As an example, we show how the most spatially variable gene varies
across the tissue.

\begin{Shaded}
\begin{Highlighting}[]
\KeywordTok{plotVisium}\NormalTok{(hbc, }\DataTypeTok{y\_reverse =} \OtherTok{FALSE}\NormalTok{, }\DataTypeTok{fill =}\NormalTok{ gnr1, }\DataTypeTok{palette=}\StringTok{"red"}\NormalTok{)}
\end{Highlighting}
\end{Shaded}

\begin{figure}
\includegraphics[width=1\linewidth,]{bioccb_files/figure-latex/plotvisium-1} \caption{Spatial expression of a highly variable gene}\label{fig:plotvisium}
\end{figure}

Finally, we show an example of an in-situ ST technology, by visualizing a breast
cancer sample assayed with the 10X Genomics Xenium platform.

\begin{Shaded}
\begin{Highlighting}[]
\KeywordTok{library}\NormalTok{(SpatialFeatureExperiment)}
\KeywordTok{library}\NormalTok{(SFEData)}
\NormalTok{jbr =}\StringTok{ }\KeywordTok{JanesickBreastData}\NormalTok{(}\StringTok{"rep1"}\NormalTok{)}
\NormalTok{jbr}
\CommentTok{\#\# class: SpatialFeatureExperiment }
\CommentTok{\#\# dim: 541 167782 }
\CommentTok{\#\# metadata(1): Samples}
\CommentTok{\#\# assays(1): counts}
\CommentTok{\#\# rownames(541): ABCC11 ACTA2 ... BLANK\_0497 BLANK\_0499}
\CommentTok{\#\# rowData names(6): ID Symbol ... vars cv2}
\CommentTok{\#\# colnames: NULL}
\CommentTok{\#\# colData names(10): Sample Barcode ... nCounts nGenes}
\CommentTok{\#\# reducedDimNames(0):}
\CommentTok{\#\# mainExpName: NULL}
\CommentTok{\#\# altExpNames(0):}
\CommentTok{\#\# spatialCoords names(2) : x\_centroid y\_centroid}
\CommentTok{\#\# imgData names(1): sample\_id}
\CommentTok{\#\# }
\CommentTok{\#\# unit:}
\CommentTok{\#\# Geometries:}
\CommentTok{\#\# colGeometries: centroids (POINT), cellSeg (POLYGON), nucSeg (GEOMETRY) }
\CommentTok{\#\# }
\CommentTok{\#\# Graphs:}
\CommentTok{\#\# sample01:}
\end{Highlighting}
\end{Shaded}

We can leverage the nature of in-situ data to explore the cell density across the
tissue, identifying the tissue's macrostructure, and the cell segmentation,
zooming in on a small portion of the tissue.

\begin{Shaded}
\begin{Highlighting}[]
\KeywordTok{library}\NormalTok{(Voyager)}
\CommentTok{\#\# 1/4 packages newly attached/loaded, see sessionInfo() for details.}
\NormalTok{cellbins \textless{}{-}}\StringTok{ }\KeywordTok{plotCellBin2D}\NormalTok{(jbr, }\DataTypeTok{hex =} \OtherTok{TRUE}\NormalTok{)}
\NormalTok{cellgeo \textless{}{-}}\StringTok{ }\KeywordTok{plotGeometry}\NormalTok{(jbr, }\StringTok{"cellSeg"}\NormalTok{, }\DataTypeTok{bbox=}\KeywordTok{c}\NormalTok{(}\StringTok{"xmin"}\NormalTok{=}\DecValTok{0}\NormalTok{, }\StringTok{"ymin"}\NormalTok{=}\DecValTok{4000}\NormalTok{, }\StringTok{"xmax"}\NormalTok{=}\DecValTok{1000}\NormalTok{, }\StringTok{"ymax"}\NormalTok{=}\DecValTok{5000}\NormalTok{))}

\KeywordTok{library}\NormalTok{(gridExtra)}
\CommentTok{\#\# 1/0 packages newly attached/loaded, see sessionInfo() for details.}
\KeywordTok{grid.arrange}\NormalTok{(cellbins, cellgeo, }\DataTypeTok{ncol=}\DecValTok{2}\NormalTok{)}
\CommentTok{\#\# Warning: Computation failed in \textasciigrave{}stat\_binhex()\textasciigrave{}}
\CommentTok{\#\# Caused by error in \textasciigrave{}compute\_group()\textasciigrave{}:}
\CommentTok{\#\# ! The package "hexbin" is required for \textasciigrave{}stat\_binhex()\textasciigrave{}}
\end{Highlighting}
\end{Shaded}

\begin{figure}
\includegraphics[width=1\linewidth,]{bioccb_files/figure-latex/plotvoyager-1} \caption{Cell density and cell boundaries of a Xenium breast cancer sample}\label{fig:plotvoyager}
\end{figure}

Finally, we can visualize the expression of marker genes after log-normalizing
the data.

\begin{Shaded}
\begin{Highlighting}[]
\NormalTok{jbr \textless{}{-}}\StringTok{ }\NormalTok{jbr[, jbr}\OperatorTok{$}\NormalTok{nCounts }\OperatorTok{\textgreater{}=}\StringTok{ }\DecValTok{20}\NormalTok{]}
\NormalTok{jbr \textless{}{-}}\StringTok{ }\KeywordTok{logNormCounts}\NormalTok{(jbr)}
\KeywordTok{library}\NormalTok{(scattermore)}
\CommentTok{\#\# 1/0 packages newly attached/loaded, see sessionInfo() for details.}
\NormalTok{strom \textless{}{-}}\StringTok{ }\KeywordTok{plotSpatialFeature}\NormalTok{(jbr, }\StringTok{"POSTN"}\NormalTok{, }\DataTypeTok{colGeometryName =} \StringTok{"centroids"}\NormalTok{,}
                   \DataTypeTok{scattermore =} \OtherTok{TRUE}\NormalTok{, }\DataTypeTok{ncol =} \DecValTok{2}\NormalTok{, }\DataTypeTok{pointsize =} \FloatTok{0.5}\NormalTok{) }\OperatorTok{+}
\StringTok{  }\KeywordTok{ggtitle}\NormalTok{(}\StringTok{"POSTN, stromal"}\NormalTok{)}

\NormalTok{fasn \textless{}{-}}\StringTok{ }\KeywordTok{plotSpatialFeature}\NormalTok{(jbr, }\StringTok{"FASN"}\NormalTok{, }\DataTypeTok{colGeometryName =} \StringTok{"centroids"}\NormalTok{,}
                   \DataTypeTok{scattermore =} \OtherTok{TRUE}\NormalTok{, }\DataTypeTok{ncol =} \DecValTok{2}\NormalTok{, }\DataTypeTok{pointsize =} \FloatTok{0.5}\NormalTok{) }\OperatorTok{+}
\StringTok{  }\KeywordTok{ggtitle}\NormalTok{(}\StringTok{"FASN, invasive"}\NormalTok{)}

\KeywordTok{grid.arrange}\NormalTok{(strom, fasn, }\DataTypeTok{ncol=}\DecValTok{2}\NormalTok{)}
\end{Highlighting}
\end{Shaded}

\begin{figure}
\includegraphics[width=1\linewidth,]{bioccb_files/figure-latex/sfemark-1} \caption{Spatial expression of marker genes}\label{fig:sfemark}
\end{figure}



\section{Components and processes for introducing new data, analytic tools, documents}\label{class}}

\subsection{Contributions and review}\label{contributions-and-review}

Proposed contributions to Bioconductor's ecosystem of software packages,
data resources, and documentation are registered at
\begin{verbatim}
https://github.com/bioconductor/contributions/issues 
\end{verbatim}
Contributors
identify a public github.com repository that houses
their software, or some durable open data repository
for a data contribution. The contributor
provides schematized information on format, licensing, and commitment
to maintenance of the contributed resource. After a series of
automated and manual verification steps, the contributed
resource enters the review process.

An example under review in December 2023 is the ``methodical''
package, submitted 27 September 2023. The issue number
at the contributions site is 3169. This contribution is of
particular interest as it addresses new data resources from
whole genome and reduced representation bisulfite sequencing
experiments. Specifics on these high-resolution studies
of DNA methylation
in a variety of clinical situtions are given below.

\subsection{Data structures}\label{data-structures}

Inheritance is a key feature of object-oriented programming (OOP) that allows us to define a new class out of existing classes and add new features, which provides reusability of code. Inheritance carries over attributes and methods defined for base classes; `Attributes' are variables that are bound in a class. They are used to define behavior and methods for objects of that class. `Methods' are functions defined within a class that receive an instance of the class, conventionally called self, as the first argument. The attributes defined for a base class will automatically be present in the derived class, and the methods for the base class will work for the derived class. The R programming language has three different class systems: S3, S4, and Reference. Inheritance in S3 classes does not have any fixed definition, and hence attributes of S3 objects can be arbitrary. Derived classes, however, inherit the methods defined for the base class. Inheritance in S4 classes is more structured, and derived classes inherit both attributes and methods of the parent class. Reference classes are similar to S4 classes, but they are mutable and have reference semantics.

S4 classes are used extensively in Bioconductor to create data structures that store complex information, such as biological assay data and metadata, in one or more slots. The entire structure can then be assigned to an R object, and the types of information in each slot of the object are tightly controlled. S4 generics and methods define functions that can be applied to these objects, providing a rich software development infrastructure while ensuring interoperability, reusability, and efficiency.

Bioconductor have established Bioconductor classes to represent different types of biological data. Data and tools distributed through Bioconductor adopt Bioconductor classes, providing convenient methods and improving usability and interoperability within the Bioconductor ecosystem.

\begin{table}
\caption{Overview of key datatypes and associated classes in Bioconductor.}
\begin{tabular}[t]{ll}
\toprule
Data Types & Bioconductor Classes\\
\midrule
Genomic coordinates (1-based, closed interval) & GRanges\\
Groups of genomic coordinates & GRangesList\\
Ragged genomic coordinates & RaggedExperiment\\
Gene sets & GeneSet\\
Rectangular Features x samples & SummarizedExperiment\\
\addlinespace
Multi-omics data & MultiAssayExperiment\\
Single-cell data & SingleCellExperiment\\
Spatial Transcriptomics & SpatialExperiment\\
Mass spectrometry data & Spectra\\
\bottomrule
\end{tabular}
\end{table}

The GRanges class represents a collection of genomic ranges and associated annotations. Each element in the vector represents a set genomic ranges in terms of the sequence name (seqnames, typically the chromosome), start and end coordinates (ranges, as an IRanges object), strand (strand, either positive, negative, or unstranded), and optional metadata columns (e.g., exon\_id and exon\_name in the below).

\begin{verbatim}
GRanges object with 4 ranges and 2 metadata columns:
      seqnames            ranges strand |   exon_id       exon_name
         <Rle>         <IRanges>  <Rle> | <integer>     <character>
  [1]        X 99883667-99884983      - |    667145 ENSE00001459322
  [2]        X 99885756-99885863      - |    667146 ENSE00000868868
  [3]        X 99887482-99887565      - |    667147 ENSE00000401072
  [4]        X 99887538-99887565      - |    667148 ENSE00001849132
  -------
  seqinfo: 722 sequences (1 circular) from an unspecified genome
\end{verbatim}

The GRangesList object serves as a container for genomic features consisting of multiple
ranges that are grouped by a parent features, such as spliced transcripts that are
comprised of exons. A GRangesList object behaves like a list and many of the same
methods for GRanges objects are available for GRangesList object as well.

The SummarizedExperiment class (see Figure \ref{fig:sesc} is a matrix-like container, where rows represent features of interest (e.g., genes, transcripts, exons, etc.) and columns represent samples. The attributes of this object include experimental results (in assays), information on observations (in rowData) and samples (in colData), and additional metadata (in metadata). SummarizedExperiment objects can simultaneouly manage several experimental results as long as they are of the same dimensions. The best benefit of using SummarizedExperiment class is the coordination of the metadata and assays when subsetting. SummarizedExperiment is similar to the historical ExpressionSet class, but more flexible in its row information, allowing both GRanges and DataFrames. ExpressionSet object can be easily converted to SummarizedExperiment.

RangedSummarizedExperiment inherits the SummarizedExperiment class, with the extended capability of storing genomic ranges (as a GRanges or GRangesList object) of interest instead of a DataFrame (S4-class objectcs similar to data.frame) of features in rows.

The MultiAssayExperiment class (presented above in
Figure \ref{fig:masc}) is modeled after the SummarizedExperiment class.
A MultiAssayExperiment instance \texttt{M} can be
filtered as a three-dimensional array.
When \texttt{G} is a vector of feature identifiers,
\texttt{C} a vector of sample identifiers, and \texttt{E} a
vector of experiment names, then \texttt{M{[}G, C, E{]}} is
a MultiAssayExperiment with content restricted to the
requested features, samples, and experiments. The MultiAssayExperiment
package includes tooling to convert data content to ``long'' or
``wide'' formats. In long format, each element of the assay array occupies
a row, accompanied by metadata associated with the element.
In wide format, each sample occupies a row, accompanied by all
assocated assay and metadata elements.

\subsection{Out-of-memory data representation strategies}\label{out-of-memory-data-representation-strategies}

We return to the ``methodical'' package
submission mentioned above.
A number of whole-genome bisulfite sequencing experiments on
tumors from various anatomic sites are available
in ExperimentHub.
Metadata in that package shows that the datasets
are large, ranging from 2-40 gigabytes. One smaller
dataset is provided for illustration.

%\begin{Shaded}
%\begin{Highlighting}[]
%\KeywordTok{library}\NormalTok{(TumourMethData)}
%\NormalTok{demm =}\StringTok{ }\KeywordTok{download\_meth\_dataset}\NormalTok{(}\StringTok{"mcrpc\_wgbs\_hg38\_chr11"}\NormalTok{)}
%\CommentTok{\#\# [1] "A HDF5 SummarizedExperiment is already present in /home/vincent/TEMP/RtmpIQh7nv/mcrpc\_wgbs\_hg38\_chr11 and is being returned"}
%\NormalTok{demm}
%\CommentTok{\#\# class: RangedSummarizedExperiment }
%\CommentTok{\#\# dim: 1333114 100 }
%\CommentTok{\#\# metadata(5): genome is\_h5 ref\_CpG chrom\_sizes descriptive\_stats}
%\CommentTok{\#\# assays(2): beta cov}
%\CommentTok{\#\# rownames: NULL}
%\CommentTok{\#\# rowData names(0):}
%\CommentTok{\#\# colnames(100): DTB\_003 DTB\_005 ... DTB\_265 DTB\_266}
%\CommentTok{\#\# colData names(4): metastatis\_site subtype age sex}
%\KeywordTok{rowRanges}\NormalTok{(demm)}
%\CommentTok{\#\# GRanges object with 1333114 ranges and 0 metadata columns:}
%\CommentTok{\#\#             seqnames    ranges strand}
%\CommentTok{\#\#                \textless{}Rle\textgreater{} \textless{}IRanges\textgreater{}  \textless{}Rle\textgreater{}}
%\CommentTok{\#\#         [1]    chr11     60077      *}
%\CommentTok{\#\#         [2]    chr11     60088      *}
%\CommentTok{\#\#         [3]    chr11     60365      *}
%\CommentTok{\#\#         [4]    chr11     60941      *}
%\CommentTok{\#\#         [5]    chr11     60979      *}
%\CommentTok{\#\#         ...      ...       ...    ...}
%\CommentTok{\#\#   [1333110]    chr11 135076482      *}
%\CommentTok{\#\#   [1333111]    chr11 135076496      *}
%\CommentTok{\#\#   [1333112]    chr11 135076502      *}
%\CommentTok{\#\#   [1333113]    chr11 135076507      *}
%\CommentTok{\#\#   [1333114]    chr11 135076510      *}
%\CommentTok{\#\#   {-}{-}{-}{-}{-}{-}{-}}
%\CommentTok{\#\#   seqinfo: 25 sequences from an unspecified genome; no seqlengths}
%\KeywordTok{names}\NormalTok{(}\KeywordTok{colData}\NormalTok{(demm))}
%\CommentTok{\#\# [1] "metastatis\_site" "subtype"         "age"             "sex"}
%\KeywordTok{table}\NormalTok{(demm}\OperatorTok{$}\NormalTok{metastatis\_site)}
%\CommentTok{\#\# }
%\CommentTok{\#\#       Bone      Liver Lymph\_node      Other }
%\CommentTok{\#\#         43         11         38          8}
%\end{Highlighting}
%\end{Shaded}

\begin{shaded}
\begin{verbatim}
library(TumourMethData)
demm = download_meth_dataset("mcrpc_wg ..." ... [TRUNCATED] 
demm
## class: RangedSummarizedExperiment 
## dim: 1333114 100 
## metadata(5): genome is_h5 ref_CpG chrom_sizes descriptive_stats
## assays(2): beta cov
## rownames: NULL
## rowData names(0):
## colnames(100): DTB_003 DTB_005 ... DTB_265 DTB_266
## colData names(4): metastatis_site subtype age sex
rowRanges(demm)
## GRanges object with 1333114 ranges and 0 metadata columns:
##             seqnames    ranges strand
##                <Rle> <IRanges>  <Rle>
##         [1]    chr11     60077      *
##         [2]    chr11     60088      *
##         [3]    chr11     60365      *
##         [4]    chr11     60941      *
##         [5]    chr11     60979      *
##         ...      ...       ...    ...
##   [1333110]    chr11 135076482      *
##   [1333111]    chr11 135076496      *
##   [1333112]    chr11 135076502      *
##   [1333113]    chr11 135076507      *
##   [1333114]    chr11 135076510      *
##   -------
##   seqinfo: 25 sequences from an unspecified genome; no seqlengths
names(colData(demm))
## [1] "metastatis_site" "subtype"         "age"             "sex"            
table(demm$metastatis_site)
##      Bone      Liver Lymph_node      Other 
##        43         11         38          8 
\end{verbatim}
\end{shaded}


References to \texttt{demm} involve an 800MB excerpt of a
prostate cancer atlas with a
storage footprint of 40GB.
Ideally,
queries about particular genomic
regions on particular samples, whole-sample statistical summaries,
and searches for patterns can be carried out without
specific accommodation of the data size or representation.
The DelayedArray package helps pursue this aim. We'll illustrate
by interrogating the prostate cancer WGBS data for ``beta''
(fraction of locus that is methylated) values in the vicinity of
gene ATM.

\begin{shaded}
\begin{verbatim}
library(EnsDb.Hsapiens.v86)
gg = genes(EnsDb.Hsapiens.v86)
# get gene addresses
atmpos = gg[gg$gene_name == "ATM" &
gg$gene_biotype == "protein_coding"] # filter to ATM
seqlevelsStyle(atmpos) = "UCSC"
assay(subsetByOverlaps(demm, atmpos+1e6))
## <18110 x 100> DelayedMatrix object of type "double":
##          DTB_003 DTB_005 DTB_008 ... DTB_265 DTB_266
##     [1,]  0.1053  0.7660  0.9206   .  0.6944  0.9412
##     [2,]  0.4062  0.9091  0.9318   .  0.5676  1.0000
##     [3,]  0.1379  0.0000  0.7400   .  0.4643  0.9231
##     [4,]  0.2308  0.9231  0.9149   .  0.8929  0.9286
##     [5,]  0.1481  0.8500  0.8864   .  0.8710  0.9762
##      ...       .       .       .   .       .       .
## [18106,]  0.4138  0.3143  0.3208   . 0.17647 0.10000
## [18107,]  0.2727  0.2745  0.4143   . 0.22500 0.32500
## [18108,]  0.2258  0.4800  0.5775   . 0.08889 0.25000
## [18109,]  0.5278  0.7059  0.8088   . 0.55263 0.97561
## [18110,]  0.2778  0.3137  0.6957   . 0.52632 0.35714
\end{verbatim}
\end{shaded}


The numeric values presented above are just the
``corners'' of the associated array, presented as a ``check''
on the content requested. Transfer of array content to
the CPU for numerical analysis only occurs on demand,
which can be tailored to the quantity of RAM available
at analysis time.

\subsection{Quality assessment of Bioconductor resources}\label{quality-assessment-of-bioconductor-resources}

Figure \ref{fig:qapic} is an overview of the periodic ecosystem
testing process for Bioconductor software packages in the
release branch. All Bioconductor
and CRAN packages on which they depend are present and are updated
on change to sources.

\begin{figure}
\includegraphics[width=1.21\linewidth,]{QApage} \caption{Build report for Bioc 3.18, 12-29-2023.}\label{fig:qapic}
\end{figure}

The project distributes source tarballs for Linux-like systems, and
compiled binaries for MacOS and Windows. Numbers in red boxes indicate
failures to install, build, or check. Failure events are
frequently platform-specific; full logs are provided
on the build report pages to help developers isolate and fix
build and check errors. When failures are persistent, developers
are contacted by core. If contact cannot be made and failures
continue, packages are deprecated for at least one release, and
then removed.

\section{Pedagogics and workforce development}\label{pedagogics-and-workforce-development}

The Bioconductor project has undertaken a number
of initiatives to support growth of the
scientific workforce's capacity to efficiently
integrate and interpret
genome-scale experiments.

\begin{itemize}
\item
  \textbf{Partnering with The Carpentries.} The Carpentries \url{https://carpentries.org} is a non-profit organization focused on teaching programming
  and data science to researchers. The organization defines ``good
  practices in lesson design and development, and open source
  collaboration skills''. Bioconductor community members have
  created bioc-intro, bioc-project, and bioc-rnaseq repositories
  using The Carpentries Incubator template. This arrangement helps
  Bioconductor create and manage a ``train the trainer'' process
  according to tested pedagogical principles.
\item
  \textbf{Curating monographs for topics in genomic data science.} The
  breadth of Bioconductor resources for genomics, combined with the
  energetic approach to software and annotation upkeep in the project,
  empowers Bioconductor developers to produce unified, wide-ranging,
  computable documents on topics of interest to the broader
  cancer genomics community. Books currently available
  at bioconductor.org include OSCA (Orchestrating Single Cell Analysis
  with Bioconductor), SingleRBook (Assigning cell types with SingleR),
  csawBook (Analysis of ChIP-seq data), OHCA (Orchestrating Hi-C
  Analysis with Bioconductor) and R for Mass Spectrometry. Very
  recently, Jacques Serizay of Institut Pasteur has contributed
  a book authoring framework called BiocBook. This transforms documents
  marked up in Posit's quarto format into web-based books backed up by Docker
  containers and maintained with templated GitHub actions. The
  OHCA book is produced and managed with BiocBook.
\item
  \textbf{A system for authoring and deploying interactive workshops.}
\end{itemize}

Figure \ref{fig:wssc} gives an overview of the resources and
objectives of the system underlying \url{workshop.bioconductor.org}.
Given a kubernetes-enabled cluster
the workshop system assembles

\begin{itemize}
\tightlist
\item
  compute and storage elements,
\item
  static components (training texts and shareable data),
\item
  development environments (containers with all runtime elements
  required to compiled code, conduct analyses, communicate with GPUs).
\end{itemize}

\begin{figure}
\includegraphics[width=0.8\linewidth,]{WorkshopSCHEMA} \caption{Workshop.bioconductor.org schematic.}\label{fig:wssc}
\end{figure}

A lightly customized deployment of the Galaxy system (usegalaxy.org)
is used to deal with authentication
and process initiation and termination.

This system has been used to serve interactive workshops in a number
of international conferences. Content in R markdown or quarto
can be produced by anyone interested in offering a workshop, and
the ``BiocWorkshopSubmit'' app at workshop.bioconductor.org
can be used to identify new content to the system. Markdown
documents will be analyzed to determine what resources are needed
for the containerization of workshop software and data components,
and the container will be created and registered at the GitHub
Container Registry. Arrangements to deploy the workshop over
a given calendar period can be made with Bioconductor core. The
workshop container can be used to conduct the workshop on any
system with a Docker client.

\section{Conclusions and paths forward}\label{conclusions-and-paths-forward}

We have described several aspects of
Bioconductor's approach to ecosystem management for cancer
genomics data science resources. In light of
the dynamism
of biotechnological innovation, it is clear that the project
must anticipate change. But it is challenging to introduce
changes to processes on which a very large community depends
for their daily research work. Commitments to supporting reproducible
research entail that Bioconductor preserves decades worth of images
of software and data for immediate retrieval via
web request by parties unknown
to the project.

We'll conclude this report with a few observations on
general paths that the project is likely to take that
should have favorable consequences to researchers in
cancer genomics.

\begin{itemize}
\item
  \textbf{Language-agnostic data and annotation} The \texttt{alabaster.*} packages
  introduced in Bioconductor 3.17 are designed to convert existing
  Bioconductor data structures to formats that are more readily ingested
  by software in other languages. Thus the \texttt{alabaster.mae}
  package will convert a MultiAssayExperiment into a collection
  of files of metadata (serialized in JSON), sample-level data
  (serialized as CSV), and assay data (serialized to HDF5).
\item
  \textbf{Zero-configuration genomic analysis environments} Users
  of Docker containers have long been able to take advantage of
  Bioconductor containers pre-populated with Rstudio and runtime
  resources to support installation of any desired software packages.
  The bioc2u system (\url{https://github.com/bioconductor/bioc2u}) in conjunction
  with r2u (\url{github.com/eddelbuettel/r2u}) introduces the
  availability of Debian packages for all Bioconductor packages,
  made available in a CRAN-like repository. Given a system running
  Ubuntu 22 or 20, the apt package manager will resolve any package
  requests with tested, fully linked binary packages. Users do not
  have to perform any configuration or compilation of system
  utilities or package code. This practice can greatly reduce
  resource consumption that occurs when individuals or
  workflow systems need to compile
  every package and its dependencies to perform analyses.
\item
  \textbf{Computation at the data} Several members of Bioconductor's
  development core are on the technical development team of
  NHGRI's Analysis and Visualization Laboratory (AnVIL). The aim
  of this project is to overthrow the prevalent model of downloading data for
  local analysis. AnVIL mobilizes commercial cloud computing and
  storage to support truly elastic genomic analysis -- create and
  pay for only the computation you need. The basic
  strategy is described in Schatz et al. (\protect\hyperlink{ref-Schatz2022}{2022}). This system was
  used in the production of the Telomere-to-Telomere
  genome build, see Aganezov et al. (\protect\hyperlink{ref-Aganezov2022}{2022}).
\end{itemize}

We hope that the project can continue to support researchers in cancer
genomics for another 20 years!

\section{Acknowledgments}\label{acknowledgments}

This work was supported in part by NIH NCI 3U24CA180996-10S1, NHGRI 5U24HG004059-18, and NSF ACCESS allocation BIR190004.


\pagebreak

%\section{Software packages used in the construction of Figure \ref{fig:easfin}}\label{app3}

\section{Figure 7 software}\label{app3}

\begin{tabular}{llll}
Package & Version & Date(UTC) & Source\\
\hline
abind & 1.4-5 & 2016-07-21 & RSPM (R 4.2.0)\\
AnnotationDbi & 1.64.1 & 2023-11-03 & Bioconductor\\
AnnotationHub & 3.10.0 & 2023-10-24 & Bioconductor\\
backports & 1.4.1 & 2021-12-13 & RSPM (R 4.2.0)\\
bcellViper & 1.38.0 & 2023-10-26 & Bioconductor\\
\addlinespace
Biobase & 2.62.0 & 2023-10-24 & Bioconductor\\
BiocFileCache & 2.10.1 & 2023-10-26 & Bioconductor\\
BiocGenerics & 0.48.1 & 2023-11-01 & Bioconductor\\
BiocManager & 1.30.22 & 2023-08-08 & RSPM (R 4.2.0)\\
BiocParallel & 1.36.0 & 2023-10-24 & Bioconductor\\
\addlinespace
BiocVersion & 3.18.0 & 2023-04-25 & Bioconductor\\
Biostrings & 2.70.1 & 2023-10-25 & Bioconductor\\
bit & 4.0.5 & 2022-11-15 & RSPM (R 4.2.0)\\
bit64 & 4.0.5 & 2020-08-30 & RSPM (R 4.2.0)\\
bitops & 1.0-7 & 2021-04-24 & RSPM (R 4.2.0)\\
\addlinespace
blob & 1.2.4 & 2023-03-17 & RSPM (R 4.2.0)\\
broom & 1.0.5 & 2023-06-09 & RSPM (R 4.2.0)\\
bspm & 0.5.5 & 2023-08-22 & CRAN (R 4.3.1)\\
cachem & 1.0.8 & 2023-05-01 & RSPM (R 4.2.0)\\
car & 3.1-2 & 2023-03-30 & RSPM (R 4.2.0)\\
\addlinespace
carData & 3.0-5 & 2022-01-06 & RSPM (R 4.2.0)\\
class & 7.3-22 & 2023-05-03 & RSPM (R 4.2.0)\\
cli & 3.6.2 & 2023-12-11 & RSPM (R 4.3.0)\\
codetools & 0.2-19 & 2023-02-01 & RSPM (R 4.2.0)\\
coin & 1.4-3 & 2023-09-27 & RSPM (R 4.3.0)\\
\addlinespace
colorspace & 2.1-0 & 2023-01-23 & RSPM (R 4.2.0)\\
cowplot & 1.1.2 & 2023-12-15 & RSPM (R 4.3.0)\\
crayon & 1.5.2 & 2022-09-29 & RSPM (R 4.2.0)\\
curl & 5.2.0 & 2023-12-08 & RSPM (R 4.3.0)\\
%data.table & 1.14.10 & 2023-12-08 & RSPM (R 4.3.0)\\
\addlinespace
DBI & 1.1.3 & 2022-06-18 & RSPM (R 4.2.0)\\
dbplyr & 2.4.0 & 2023-10-26 & RSPM (R 4.3.0)\\
decoupleR & 2.8.0 & 2023-10-24 & Bioconductor\\
DelayedArray & 0.28.0 & 2023-10-24 & Bioconductor\\
DESeq2 & 1.42.0 & 2023-10-24 & Bioconductor\\
\addlinespace
digest & 0.6.33 & 2023-07-07 & RSPM (R 4.2.0)\\
dorothea & 1.14.0 & 2023-10-26 & Bioconductor\\
dplyr & 1.1.4 & 2023-11-17 & RSPM (R 4.3.0)\\
e1071 & 1.7-14 & 2023-12-06 & RSPM (R 4.3.0)\\
easier & 1.8.0 & 2023-10-24 & Bioconductor\\
\addlinespace
easierData & 1.8.0 & 2023-10-26 & Bioconductor\\
ellipsis & 0.3.2 & 2021-04-29 & RSPM (R 4.2.0)\\
evaluate & 0.23 & 2023-11-01 & RSPM (R 4.3.0)\\
ExperimentHub & 2.10.0 & 2023-10-24 & Bioconductor\\
fansi & 1.0.6 & 2023-12-08 & RSPM (R 4.3.0)\\
\end{tabular}
\pagebreak

\begin{tabular}{llll}
farver & 2.1.1 & 2022-07-06 & RSPM (R 4.2.0)\\
fastmap & 1.1.1 & 2023-02-24 & RSPM (R 4.2.0)\\
filelock & 1.0.3 & 2023-12-11 & RSPM (R 4.3.0)\\
generics & 0.1.3 & 2022-07-05 & RSPM (R 4.2.0)\\
GenomeInfoDb & 1.38.1 & 2023-11-08 & Bioconductor\\
\addlinespace
GenomeInfoDbData & 1.2.11 & <NA> & Bioconductor\\
GenomicRanges & 1.54.1 & 2023-10-29 & Bioconductor\\
ggplot2 & 3.4.4 & 2023-10-12 & RSPM (R 4.3.0)\\
ggpubr & 0.6.0 & 2023-02-10 & RSPM (R 4.2.0)\\
ggrepel & 0.9.4 & 2023-10-13 & RSPM (R 4.3.0)\\
\addlinespace
ggsignif & 0.6.4 & 2022-10-13 & RSPM (R 4.2.0)\\
glue & 1.6.2 & 2022-02-24 & RSPM (R 4.2.0)\\
gridExtra & 2.3 & 2017-09-09 & RSPM (R 4.2.0)\\
gtable & 0.3.4 & 2023-08-21 & RSPM (R 4.2.0)\\
htmltools & 0.5.7 & 2023-11-03 & RSPM (R 4.3.0)\\
\addlinespace
htmlwidgets & 1.6.4 & 2023-12-06 & RSPM (R 4.3.0)\\
httpuv & 1.6.13 & 2023-12-06 & RSPM (R 4.3.0)\\
httr & 1.4.7 & 2023-08-15 & RSPM (R 4.2.0)\\
interactiveDisplayBase & 1.40.0 & 2023-10-24 & Bioconductor\\
IRanges & 2.36.0 & 2023-10-24 & Bioconductor\\
\addlinespace
jsonlite & 1.8.8 & 2023-12-04 & RSPM (R 4.3.0)\\
KEGGREST & 1.42.0 & 2023-10-24 & Bioconductor\\
kernlab & 0.9-32 & 2023-01-31 & RSPM (R 4.2.0)\\
KernSmooth & 2.23-22 & 2023-07-10 & RSPM (R 4.2.0)\\
knitr & 1.45 & 2023-10-30 & RSPM (R 4.3.0)\\
\addlinespace
labeling & 0.4.3 & 2023-08-29 & RSPM (R 4.2.0)\\
later & 1.3.2 & 2023-12-06 & RSPM (R 4.3.0)\\
lattice & 0.22-5 & 2023-10-24 & RSPM (R 4.3.0)\\
lazyeval & 0.2.2 & 2019-03-15 & RSPM (R 4.2.0)\\
libcoin & 1.0-10 & 2023-09-27 & RSPM (R 4.3.0)\\
\addlinespace
lifecycle & 1.0.4 & 2023-11-07 & RSPM (R 4.3.0)\\
limSolve & 1.5.7 & 2023-09-21 & RSPM (R 4.3.0)\\
locfit & 1.5-9.8 & 2023-06-11 & RSPM (R 4.2.0)\\
lpSolve & 5.6.20 & 2023-12-10 & RSPM (R 4.3.0)\\
magrittr & 2.0.3 & 2022-03-30 & RSPM (R 4.2.0)\\
\addlinespace
MASS & 7.3-60 & 2023-05-04 & RSPM (R 4.2.0)\\
Matrix & 1.6-4 & 2023-11-30 & RSPM (R 4.3.0)\\
MatrixGenerics & 1.14.0 & 2023-10-24 & Bioconductor\\
matrixStats & 1.2.0 & 2023-12-11 & RSPM (R 4.3.0)\\
memoise & 2.0.1 & 2021-11-26 & RSPM (R 4.2.0)\\
\addlinespace
mime & 0.12 & 2021-09-28 & RSPM (R 4.2.0)\\
mixtools & 2.0.0 & 2022-12-05 & RSPM (R 4.2.0)\\
modeltools & 0.2-23 & 2020-03-05 & RSPM (R 4.2.0)\\
multcomp & 1.4-25 & 2023-06-20 & RSPM (R 4.2.0)\\
munsell & 0.5.0 & 2018-06-12 & RSPM (R 4.2.0)\\
\end{tabular}
\pagebreak

\begin{tabular}{llll}
\addlinespace
mvtnorm & 1.2-4 & 2023-11-27 & RSPM (R 4.3.0)\\
nlme & 3.1-164 & 2023-11-27 & RSPM (R 4.3.0)\\
pillar & 1.9.0 & 2023-03-22 & RSPM (R 4.2.0)\\
pkgconfig & 2.0.3 & 2019-09-22 & RSPM (R 4.2.0)\\
plotly & 4.10.3 & 2023-10-21 & RSPM (R 4.3.0)\\
\addlinespace
plyr & 1.8.9 & 2023-10-02 & RSPM (R 4.3.0)\\
png & 0.1-8 & 2022-11-29 & RSPM (R 4.2.0)\\
preprocessCore & 1.64.0 & 2023-10-24 & Bioconductor\\
progeny & 1.24.0 & 2023-10-24 & Bioconductor\\
promises & 1.2.1 & 2023-08-10 & RSPM (R 4.2.0)\\
\addlinespace
proxy & 0.4-27 & 2022-06-09 & RSPM (R 4.2.0)\\
purrr & 1.0.2 & 2023-08-10 & RSPM (R 4.2.0)\\
quadprog & 1.5-8 & 2019-11-20 & RSPM (R 4.2.0)\\
quantiseqr & 1.10.0 & 2023-10-24 & Bioconductor\\
R6 & 2.5.1 & 2021-08-19 & RSPM (R 4.2.0)\\
\addlinespace
rappdirs & 0.3.3 & 2021-01-31 & RSPM (R 4.2.0)\\
Rcpp & 1.0.11 & 2023-07-06 & RSPM (R 4.2.0)\\
RCurl & 1.98-1.13 & 2023-11-02 & RSPM (R 4.3.0)\\
reshape2 & 1.4.4 & 2020-04-09 & CRAN (R 4.0.1)\\
rlang & 1.1.2 & 2023-11-04 & RSPM (R 4.3.0)\\
\addlinespace
rmarkdown & 2.25 & 2023-09-18 & RSPM (R 4.3.0)\\
ROCR & 1.0-11 & 2020-05-02 & RSPM (R 4.2.0)\\
RSQLite & 2.3.4 & 2023-12-08 & RSPM (R 4.3.0)\\
rstatix & 0.7.2 & 2023-02-01 & RSPM (R 4.2.0)\\
S4Arrays & 1.2.0 & 2023-10-24 & Bioconductor\\
\addlinespace
S4Vectors & 0.40.2 & 2023-11-23 & Bioconductor 3.18 (R 4.3.2)\\
sandwich & 3.1-0 & 2023-12-11 & RSPM (R 4.3.0)\\
scales & 1.3.0 & 2023-11-28 & RSPM (R 4.3.0)\\
segmented & 2.0-1 & 2023-12-19 & RSPM (R 4.3.0)\\
sessioninfo & 1.2.2 & 2021-12-06 & RSPM (R 4.2.0)\\
\addlinespace
shiny & 1.8.0 & 2023-11-17 & RSPM (R 4.3.0)\\
SparseArray & 1.2.2 & 2023-11-07 & Bioconductor\\
startup & 0.21.0 & 2023-12-11 & RSPM (R 4.3.0)\\
stringi & 1.8.3 & 2023-12-11 & RSPM (R 4.3.0)\\
stringr & 1.5.1 & 2023-11-14 & RSPM (R 4.3.0)\\
\addlinespace
SummarizedExperiment & 1.32.0 & 2023-10-24 & Bioconductor\\
survival & 3.5-7 & 2023-08-14 & RSPM (R 4.2.0)\\
TH.data & 1.1-2 & 2023-04-17 & RSPM (R 4.2.0)\\
tibble & 3.2.1 & 2023-03-20 & RSPM (R 4.3.0)\\
tidyr & 1.3.0 & 2023-01-24 & RSPM (R 4.2.0)\\
\addlinespace
tidyselect & 1.2.0 & 2022-10-10 & RSPM (R 4.2.0)\\
utf8 & 1.2.4 & 2023-10-22 & RSPM (R 4.3.0)\\
vctrs & 0.6.5 & 2023-12-01 & RSPM (R 4.3.0)\\
viper & 1.36.0 & 2023-10-24 & Bioconductor\\
viridisLite & 0.4.2 & 2023-05-02 & RSPM (R 4.2.0)\\
\addlinespace
withr & 2.5.2 & 2023-10-30 & RSPM (R 4.3.0)\\
xfun & 0.41 & 2023-11-01 & RSPM (R 4.3.0)\\
xtable & 1.8-4 & 2019-04-21 & RSPM (R 4.2.0)\\
XVector & 0.42.0 & 2023-10-24 & Bioconductor\\
yaml & 2.3.8 & 2023-12-11 & RSPM (R 4.3.0)\\
\addlinespace
zlibbioc & 1.48.0 & 2023-10-24 & Bioconductor\\
zoo & 1.8-12 & 2023-04-13 & RSPM (R 4.2.0)\\
\hline
\end{tabular}
\end{table}



\section{Acknowledgments}\label{acknowledgments}

This work was supported in part by NIH NCI 3U24CA180996-10S1, NHGRI 5U24HG004059-18, and NSF ACCESS allocation BIR190004.


%\input{newrefs}
\bibliographystyle{vjcunsrt}
\bibliography{biocmimb}

\end{document}
